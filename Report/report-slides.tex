%
% ---------------------------------------------------------------
% Copyright (C) 2012-2018 Gang Li
% ---------------------------------------------------------------
%
% This work is the default powerdot-tuliplab style test file and may be
% distributed and/or modified under the conditions of the LaTeX Project Public
% License, either version 1.3 of this license or (at your option) any later
% version. The latest version of this license is in
% http://www.latex-project.org/lppl.txt and version 1.3 or later is part of all
% distributions of LaTeX version 2003/12/01 or later.
%
% This work has the LPPL maintenance status "maintained".
%
% This Current Maintainer of this work is Gang Li.
%
%

\documentclass[
 size=14pt,
 paper=smartboard,  %a4paper, smartboard, screen
 mode=present, 		%present, handout, print
 display=slides, 	% slidesnotes, notes, slides
 style=tuliplab,  	% TULIP Lab style
 pauseslide,
 fleqn,leqno]{powerdot}


\usepackage{cancel}
\usepackage{caption}
\usepackage{stackengine}
\usepackage{smartdiagram}
\usepackage{attrib}
\usepackage{amssymb}
\usepackage{amsmath} 
\usepackage{amsthm} 
\usepackage{mathtools}
\usepackage{rotating}
\usepackage{graphicx}
\usepackage{boxedminipage}
\usepackage{rotate}
\usepackage{calc}
\usepackage[absolute]{textpos}
\usepackage{psfrag,overpic}
\usepackage{fouriernc}
\usepackage{pstricks,pst-3d,pst-grad,pstricks-add,pst-text,pst-node,pst-tree}
\usepackage{moreverb,epsfig,subfigure}
\usepackage{color}
\usepackage{booktabs}
\usepackage{etex}
\usepackage{breqn}
\usepackage{multirow}
\usepackage{natbib}
\usepackage{bibentry}
\usepackage{gitinfo2}
\usepackage{siunitx}
\usepackage{nicefrac}
%\usepackage{geometry}
%\geometry{verbose,letterpaper}
\usepackage{media9}
\usepackage{animate}
%\usepackage{movie15}
\usepackage{auto-pst-pdf}

\usepackage{breakurl}
\usepackage{fontawesome}
\usepackage{xcolor}
\usepackage{multicol}



\usepackage{verbatim}
\usepackage[utf8]{inputenc}
\usepackage{dtk-logos}
\usepackage{tikz}
\usepackage{adigraph}
%\usepackage{tkz-graph}
\usepackage{hyperref}
%\usepackage{ulem}
\usepackage{pgfplots}
\usepackage{verbatim}
\usepackage{fontawesome}


\usepackage{todonotes}
% \usepackage{pst-rel-points}
\usepackage{animate}
\usepackage{fontawesome}

\usepackage{listings}
\lstset{frameround=fttt,
frame=trBL,
stringstyle=\ttfamily,
backgroundcolor=\color{yellow!20},
basicstyle=\footnotesize\ttfamily}
\lstnewenvironment{code}{
\lstset{frame=single,escapeinside=`',
backgroundcolor=\color{yellow!20},
basicstyle=\footnotesize\ttfamily}
}{}


\usepackage{hyperref}
\hypersetup{ % TODO: PDF meta Data
  pdftitle={Presentation Title},
  pdfauthor={Gang Li},
  pdfpagemode={FullScreen},
  pdfborder={0 0 0}
}


% \usepackage{auto-pst-pdf}
% package to show source code

\definecolor{LightGray}{rgb}{0.9,0.9,0.9}
\newlength{\pixel}\setlength\pixel{0.000714285714\slidewidth}
\setlength{\TPHorizModule}{\slidewidth}
\setlength{\TPVertModule}{\slideheight}
\newcommand\highlight[1]{\fbox{#1}}
\newcommand\icite[1]{{\footnotesize [#1]}}

\newcommand\twotonebox[2]{\fcolorbox{pdcolor2}{pdcolor2}
{#1\vphantom{#2}}\fcolorbox{pdcolor2}{white}{#2\vphantom{#1}}}
\newcommand\twotoneboxo[2]{\fcolorbox{pdcolor2}{pdcolor2}
{#1}\fcolorbox{pdcolor2}{white}{#2}}
\newcommand\vpspace[1]{\vphantom{\vspace{#1}}}
\newcommand\hpspace[1]{\hphantom{\hspace{#1}}}
\newcommand\COMMENT[1]{}

\newcommand\placepos[3]{\hbox to\z@{\kern#1
        \raisebox{-#2}[\z@][\z@]{#3}\hss}\ignorespaces}

\renewcommand{\baselinestretch}{1.2}


\newcommand{\draftnote}[3]{
	\todo[author=#2,color=#1!30,size=\footnotesize]{\textsf{#3}}	}
% TODO: add yourself here:
%
\newcommand{\gangli}[1]{\draftnote{blue}{GLi:}{#1}}
\newcommand{\shaoni}[1]{\draftnote{green}{sn:}{#1}}
\newcommand{\gliMarker}
	{\todo[author=GLi,size=\tiny,inline,color=blue!40]
	{Gang Li has worked up to here.}}
\newcommand{\snMarker}
	{\todo[author=Sn,size=\tiny,inline,color=green!40]
	{Shaoni has worked up to here.}}

%%%%%%%%%%%%%%%%%%%%%%%%%%%%%%%%%%%%%%%%%%%%%%%%%%%%%%%%%%%%%%%%%%%%%%%%
% title
% TODO: Customize to your Own Title, Name, Address
%
\title{BIKE SHARING DEMAND PREDICTION}
\author{
Chouyin Zhang
\\
\\Hunan University
\\China
}
\date{\gitCommitterDate}


% Customize the setting of slides
\pdsetup{
% TODO: Customize the left footer, and right footer
rf=\href{http://www.tulip.org.au}{
Last Changed by: \textsc{\gitCommitterName}\ \gitVtagn-\gitAbbrevHash\ (\gitAuthorDate)
},
cf={BIKE SHARING DEMAND PREDICTION},
}


\begin{document}

\maketitle
%\begin{slide}{Overview}
%\tableofcontents[content=sections]
%\end{slide}
%%==========================================================================================
\begin{slide}[toc=,bm=]{Overview}
\tableofcontents[content=currentsection,type=1]
\end{slide}
%%
%%==========================================================================================


\section{Problem Definition}


%%==========================================================================================
%%
\begin{slide}{Problem Introduction}
\begin{center}
\twotonebox{\rotatebox{90}{Defn}}{\parbox{.86\textwidth}
{Problem Background
\begin{itemize}
\item Bike sharing systems are a means of renting bicycles where the process of obtaining membership, rental, and bike return is automated via a network of kiosk locations throughout a city. 
\item  Using these systems, people are able rent a bike from a one location and return it to a different place on an as-needed basis.
\end{itemize}
}}

\end{center}

\vspace{1ex}

\begin{center}
\twotonebox{\rotatebox{90}{Defn}}{\parbox{.86\textwidth}
{Problem introduction
\begin{itemize}
\item In the data generated, the duration of travel, departure location, arrival location, and time elapsed is explicitly recorded. 
\item  Bike sharing systems therefore function as a sensor network, which can be used for studying mobility in a city.
\item In this competition, participants are asked to combine historical usage patterns with weather data in order to \emph{forecast bike rental demand}.
\end{itemize}
	}}
	
\end{center}


%%==========================================================================================
%%==========================================================================================

\end{slide}
%%
%%==========================================================================================


\section{Exploratory Data Analysis}



%%==========================================================================================
%%

%%==========================================================================================


%%
%%==========================================================================================
\begin{slide}[toc=,bm=]{Exploratory Data Analysis}
\begin{itemize}
\item
Distribution of variables
		
\begin{itemize}
\item
Distribution of dependent variables \emph{count}, which represents bike usage.
			
\item
 Distribution of \emph{count} with \emph{season}, \emph{holiday}, \emph{weather},\emph{month}, and \emph{working day}.		
			
	
\end{itemize}
\end{itemize}
	
\begin{center}
\begin{minipage}{0.3\linewidth}
\centering

\begin{figure}
	\caption{Description of Count}
	\includegraphics[width=6.5cm,height=5.5cm]{D:/D:downloads/p1}
\end{figure}

\end{minipage}
\hfill
\begin{minipage}{0.3\linewidth}
\centering

\begin{figure}
	\caption{Description of Count and Month}
	\includegraphics[width=6.5cm,height=5.5cm]{D:/D:downloads/P2(2)}
\end{figure}		

\end{minipage}
\hfill
\begin{minipage}{0.3\linewidth}
\centering

\begin{figure}
	\caption{Description of Count and Weather}
	\includegraphics[width=6.5cm,height=5.5cm]{D:/D:downloads/p_weather}
\end{figure}


\end{minipage}
\end{center}
	
	%%==========================================================================================
	%%==========================================================================================
	
\end{slide}

%%==========================================================================================
%%

%%
%%==========================================================================================
\begin{slide}[toc=,bm=]{Exploratory Data Analysis}
	\begin{itemize}
		\item
		Distribution of variables
		
		\begin{itemize}
			\item
			Distribution of dependent variables \emph{count}, which represents bike usage.
			
			\item
			Distribution of \emph{count} with \emph{season},\emph{holiday}, \emph{weather},\emph{month}, and \emph{working day}.		
			
			
		\end{itemize}
	\end{itemize}
	
	\begin{center}
		\begin{minipage}{0.3\linewidth}
			\centering
			
\begin{figure}
	\caption[Description of Count and Workingday]{}
	\label{fig:pworking-1}
	\includegraphics[width=6.5cm,height=5.5cm]{"D:/D:downloads/p_working (1)"}
\end{figure}
			
		\end{minipage}
		\hfill
		\begin{minipage}{0.3\linewidth}
			\centering
			
\begin{figure}
	\caption[Description of Count and Season]{}
	\label{fig:pseason}
	\includegraphics[width=6.5cm,height=5.5cm]{D:/D:downloads/p_season}
\end{figure}
				
			
		\end{minipage}
		\hfill
		\begin{minipage}{0.3\linewidth}
			\centering
			
\begin{figure}
	\caption[Description of Count and Holiday]{}
	\label{fig:pholiday}
	\includegraphics[width=6.5cm,height=5.5cm]{D:/D:downloads/p_holiday}
\end{figure}
				
			
		\end{minipage}
	\end{center}
	
	%%==========================================================================================
	%%==========================================================================================
	
\end{slide}

%%==========================================================================================
%%
\begin{slide}[toc=,bm=]{Exploratory Data Analysis}
	\begin{itemize}
		\item
		Distribution of variables
		
		\begin{itemize}
			\item
		\emph{Four jointplots} showing  how daily usage varies with the continuous variables.
					
			
		\end{itemize}
	\end{itemize}
	
	\begin{center}
		\begin{minipage}{0.3\linewidth}
			\centering
			
\begin{figure}
	\caption[Total Bike usage vs Humidity]{}
	\label{fig:jointplothumidity}
	\includegraphics[width=5.5cm,height=7.5cm]{D:/D:downloads/jointplot_humidity}
\end{figure}
		
		\end{minipage}
		\hfill
		\begin{minipage}{0.3\linewidth}
			\centering
			
\begin{figure}
	\caption[Total Bike Usage vs Temperature]{}
	\label{fig:jointplottemp}
	\includegraphics[width=5.5cm,height=7.5cm]{D:/D:downloads/jointplot_temp}
\end{figure}
				
		
		\end{minipage}
		\hfill
		\begin{minipage}{0.3\linewidth}
			\centering
		
\begin{figure}
	\caption[Total Bike usage vs Wind Speed]{}
	\label{fig:jointplotwindspeed}
	\includegraphics[width=5.5cm,height=7.5cm]{D:/D:downloads/jointplot_windspeed}
\end{figure}
				
		
		\end{minipage}
	\end{center}
	
	%%==========================================================================================
	%%==========================================================================================
	
\end{slide}

%%
%%==========================================================================================
\begin{slide}[toc=,bm=]{Exploratory Data Analysis}


		
		\begin{itemize}
			\item
			 \emph{A correlation heatmap} showing the \emph{2D correlation matrix} for the features in our dataset.
			
			
		\end{itemize}

	

\begin{figure}
	\centering
	\includegraphics[width=10.5cm, height=10.5cm]{D:/D:downloads/heatmap}
	\caption{Correlation heatmap}
	\label{fig:heatmap}
\end{figure}
		
	
	

\end{slide}

%%==========================================================================================
%%

\section{Data Preprocessing}


%%==========================================================================================
%%
\begin{slide}{Data Preprocessing--Logarithmic transformation}
	\begin{center}
		\twotonebox{\rotatebox{90}{Defn}}{\parbox{.86\textwidth}
			{Logarithmic transformation
				\begin{itemize}
					\item First we can look at the distribution of count. The distribution is heavily skewed right, which is not well approximated by a normal distribution.
					\item Taking the natural log of the count data gives a distribution slightly more normally distributed, as shown above, and also in the quantile-quantile (Q-Q) plots below. 
				\end{itemize}
		}}
		
	\end{center}

	
\begin{figure}
	\centering
	\caption[Q-Q plots for total bike usage and logarithmic transformation]{}
	\label{fig:qqplot-1}
	\includegraphics[width=8.5cm, height=5.5cm]{"D:/D:downloads/QQ_plot (1)"}
\end{figure}
	
\end{slide}

%%==========================================================================================
%%
\begin{slide}{Data Preprocessing--Drop variables}
	\begin{center}
		\twotonebox{\rotatebox{90}{Defn}}{\parbox{.86\textwidth}
			{Drop some variables from our dataframe
				\begin{itemize}
					\item 
					1.Drop casual and registered as the test set does not contain data for the casual count or the registered count.
					\\2. Drop atemp since it has almost perfect correlation with temp.
					\\3.Drop all day of the month data from the training set before training our models. 
				\end{itemize}
		}}
		
	\end{center}
	
	\vspace{1ex}
		\begin{center}
		\twotonebox{\rotatebox{90}{Defn}}{\parbox{.86\textwidth}
			{Replace the target variable count with the natural logarithm of count
				\begin{itemize}
					\item 
				As the natural logarithm of count is more normally distributed, Let us replace the target variable count with  y=ln(count+1).
				\end{itemize}
		}}
		
	\end{center}

\end{slide}
	
%%==========================================================================================
%%
\begin{slide}{Data Preprocessing-- Replace and Split}
	\begin{center}
		\twotonebox{\rotatebox{90}{Defn}}{\parbox{.86\textwidth}
			{Replace the categorical data with binary dummy variables
				\begin{itemize}
					\item 
				 The categorical data is expressed as an arbitrary numerical value. Therefore we can use dummy coding, replacing the categorical data with binary dummy variables using the pandas function. 
				\end{itemize}
		}}
		
	\end{center}
	
	\vspace{0.5ex}
	
	\begin{center}
		\twotonebox{\rotatebox{90}{Defn}}{\parbox{.86\textwidth}
			{Split the labelled training set into two sets
				\begin{itemize}
					\item 
					One for training our models, and a validation set for determining the best set of hyperparameters.
				\end{itemize}
		}}
		
	\end{center}

\vspace{0.5ex}

\begin{center}
	\twotonebox{\rotatebox{90}{Defn}}{\parbox{.86\textwidth}
		{Define a prediction function
			\begin{itemize}
				\item 
				Finally, let's define a function predict that will report prediction scores for a given model.
			\end{itemize}
	}}
	
\end{center}	
	
\end{slide}


%%
%%==========================================================================================


\section{Train and Apply Models}


%%==========================================================================================
%%







%%
%%==========================================================================================


%%==========================================================================================
%%
\begin{slide}[toc=,bm=]{Train and Apply Models}
\begin{itemize}
\item
Build and apply some chosen model

\end{itemize}
\begin{table}
\setlength{\abovecaptionskip}{0pt}
\setlength{\belowcaptionskip}{10pt}
\centering
\caption{$\alpha = 4$}

\begin{tabular}{  c  |  c }
\toprule
\centering
\texttt{Models}  & \texttt{RMSLE value}  \\
\midrule
 {\{$LinearRegression$\}}       & $0.136$ \\
 {\{$Ridge$\}}                & $0.137$ \\
 {\{$Lasso$\} }                 & $0.148$ \\
 {\{$RandomForestRegressor$\}}       & $0.084$  \\
 	
 {\{$GradientBoostingRegressor$\}}      & $0.080$  \\
 	
 {\{$XGBRegressor$\} }            & $0.085$  \\
 	
 {\{$LGBMRegressor$\} }           & $0.078$  \\
\bottomrule
\end{tabular}
\end{table}
\begin{description}
	\item
	It can be observed that the LGBMRegressor model can get the best score. \\
	So, we apply this best model to test data.
	
\end{description}

%%==========================================================================================
%%==========================================================================================

\end{slide}
%%
%%==========================================================================================
\begin{slide}[toc=,bm=]{Train and Apply Models}
	\begin{itemize}
		\item
		The output of the prediction is as below
		
	\end{itemize}
	\vspace{.4cm}          
	\begin{minipage}{0.5\linewidth}
		\centering
	
\begin{figure}
	\centering
	\caption{}
	\label{fig:predicthead}
	\includegraphics[width=8.5cm, height=5.5cm]{D:/D:downloads/predict_head}
\end{figure}
		
	\end{minipage}
%%==========================================================================================	%%==========================================================================================
	
\end{slide}

%%==========================================================================================
%%

%%
%%==========================================================================================


\section{Conclusion}

%%==========================================================================================
%%
\begin{slide}[toc=,bm=]{Conclusion}
\begin{itemize}
\item
 In this data mining task, I use the dataset from Kaggle. According the problem background given, I apply EDA, data preprocessing to the dataset, and finally apply the best model to test data, successfully finish this task.

\end{itemize}

%%==========================================================================================
\begin{note}
In conclusion,
we firstly formalized the problem of
group outlying aspects mining,

Then proposed a novel method GOAM algorithm to address the problem of
group outlying aspects mining,
and the proposed method use pruning to reduce time complexity
while identifying the suitable set of outlying features for the interested group.

Thank you and any question?
\end{note}
%%==========================================================================================

\end{slide}
%%
%%==========================================================================================


%%==========================================================================================
%

%%==========================================================================================
% TODO: Contact Page
\begin{wideslide}[toc=,bm=]{Contact Information}
\centering
\vspace{\stretch{1}}
\twocolumn[
lcolwidth=0.35\linewidth,
rcolwidth=0.65\linewidth
]
{
% \centerline{\includegraphics[scale=.2]{tulip-logo.eps}}
}
{
\vspace{\stretch{1}}
Chouyin Zhang\\
Hunan University, China

}
\vspace{\stretch{1}}
\end{wideslide}

\end{document}

\endinput
